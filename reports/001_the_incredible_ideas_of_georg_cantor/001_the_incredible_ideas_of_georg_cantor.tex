% Defaults
\documentclass{article}
\usepackage{arxiv}
\usepackage[utf8]{inputenc}
\usepackage[T1]{fontenc}
\usepackage{amsmath}
\usepackage{hyperref}
\usepackage{url}
\usepackage{booktabs}
\usepackage{amsfonts}
\usepackage{nicefrac}
\usepackage{microtype}
\usepackage[ruled,vlined]{algorithm2e}

% Meta Info
\title{
The Incredible Ideas of Georg Cantor
}
\author{
  Notes by Branden Keck\\
  \texttt{bmkeck62@gmail.com} \\
}

% Remove the Pre-Print Designation and Date
\renewcommand{\undertitle}{}
\date{}

% Begin Document Contents
\begin{document}
\maketitle

% Intro information
\section{Introduction}
I have always said to myself that if I started sharing my notes online, I'd begin with Georg Cantor.  These informal notes are intended to organize thoughts and ideas about a man whose work is absolutely fascinating to me.  I cannot write much that has not already been written about Cantor (as he is an incredibly notorious figure in the world of mathematics), however I'd like to offer a perspective on my favorite mathematical idea - the Ternary Cantor set - and the incredible man behind it.

\section{The Myth, The Man, The Legend}
I have been obsessed with Cantor's ideas for s

\subsection{Section 2a}
Placeholder

\subsubsection{Section 2a.}
Placeholder

\section{All Infinities Are Infinite, But Some Infinities Are More Infinite Than Others}
So, the title of this section may make some mathematicians want to scream.

\section{Why the Cantor Set is the Coolest Thing in Mathematics}
Placeholder

\begin{thebibliography}{1}

\bibitem{kour2014real}
Grattan-Guinness, I.
\newblock {\em Towards a Biography of Georg Cantor.}
\newblock Annals of Science, vol. 27, no. 4, 1971, pp. 345–391.
\newblock doi:10.1080/00033797100203837.

\bibitem{kour2014real}
Authors
\newblock {\em Title.}
\newblock Reference
\newblock DOI

\end{thebibliography}

\end{document}
