% Defaults
\documentclass{article}
\usepackage{arxiv}
\usepackage[utf8]{inputenc}
\usepackage[T1]{fontenc}
\usepackage{amsmath}
\usepackage{hyperref}
\usepackage{url}
\usepackage{booktabs}
\usepackage{amsfonts}
\usepackage{nicefrac}
\usepackage{microtype}
\usepackage[ruled,vlined]{algorithm2e}

% Meta Info
\title{
On Georg Cantor and the Infinite
}
\author{
  Notes by Branden Keck\\
  \texttt{bmkeck62@gmail.com} \\
}

% Remove the Pre-Print Designation and Date
\renewcommand{\undertitle}{}
\date{}

% Begin Document Contents
\begin{document}
\maketitle

% Intro information
\section{Introduction}
These informal notes are intended to organize thoughts and ideas about Georg Cantor, a man whose work is absolutely fascinating to me.  Cantor is an incredibly notorious figure in the world of mathematics whose ideas about the nature of infinity defined the world of mathematics for generations to come.  Through these notes, I'd like to offer a perspective on my favorite mathematical idea - the Ternary Cantor set - and the incredible man behind it.

\section{Cantor's Life}
Cantor's life is almost as fascinating as his ideas.  First things first:  Georg Cantor was born in Saint Petersburg on 03-Mar-1845, later moving to Germany in 1856, where he lived for the vast majority of his life [1].  Cantor's life was not free from either personal struggle or family tragedy.  His academic pursuits also proved to be difficult for him as a result of the apprehension of his peers to accept his ideas.  Cantor's hardships - like those of many great men and women - acted both as hurdles to his success and defining features in the person he became.  However, while many of these tragic and frustrating events resulted in typical exploits for a career academic, a lot of Cantor's pursuits were far from ordinary.

The details of Cantor's personal life and academic career range from an obsession with proving that William Shakespeare was actually Sir Francis Bacon to frequently writing members of the Catholic church to assure them that his ideas about the infinite do not conflict with those of the church [1,2].  With respect to the latter topic, Cantor often asserted that his mathematical ideas were reaffirmation that God is the true infinite.  And, for the most part, the Catholic church was willing to accept this school of thought as long as he continued to distinguish between the actual-infinite (mathematical tranfinitism) and the absolute-infinite (God) [2].

From a mathematical perspective, Cantor's ideas about infinity attracted supporters and detractors alike.  Notably, Cantor had earned the admiration of Richard Dedekind, another major figure in the world of set theory, at a time when his work was under heavy scrutiny by Leopold Kronecker. Aside from (dis)agreement over his ideas, Cantor's interactions with great mathematicians of his time were far-reaching.  Through efforts to publish his work, Cantor had significant correspondence with Gösta Mittag-Leffler, Henri Poincaré, and Charles Hermite.  And, Cantor had frequent interactions with Felix Klein as a member of the academic community in Germany [1].

As evidenced by the depth of these associations, Cantor was alive during a truly incredible time in the history of mathematics.  Not only did Cantor interact with many mathematicians whose names are now immortalized by the ideas they inspired (i.e. "Dedekind cut", "Kronecker delta", "Poincaré conjecture", "Hermite polynomial", "Klein bottle", etc.), but Cantor's life and ideas also function as somewhat of a link between classical mathematics and the incredible advances made during the 20th century.

One fascinating consequence of Cantor's influence is the appearance of his Continuum Hypothesis on David Hilbert's infamous list of 23 Problems.  Published in 1900, these problems proved to be incredibly influential on mathematics in the 20th century.  It is said that Hilbert's presentation of the Continuum Hypothesis (alongside 9 other problems from his list) at the 2nd International Congress of Mathematicians (ICM) in 1900 was "one of the highlights of Cantor's scientific life" [3].  It is also said that Gyula Kőnig's presentation of a proof that the Continuum Hypothesis is "meaningless" because continuum cannot be well-ordered threw Cantor into an absolute frenzy at the following ICM in 1904 [1, 3].  This sequence of events is a perfect example of Cantor's influence in mathematics as well as his professional struggle to defend his work and the emotional toll it took.  It is also fascinating that Kőnig's work was later disproved by Ernst Zermelo (one half of the eponymous Zermelo-Fraenkel axiomatic set theory), adding yet another connection between Cantor and the world of mathematics [3].

There are a seemingly endless number of interesting stories involving Georg Cantor's personal life and his academic career.  However, Cantor's exploits form a finite - yet essential - subset of the events in the history of mathematics.  Cantor's ideas of infinity, though heavily debated in their time, have sculpted the landscape of modern set theory.  In the next section, Cantor's ideas will be discussed through a series of thought experiments and half-proofs to show the vast impact of his contribution to set theory.

\section{Cantor's Ideas}
As a student of engineering and mathematics, I first encountered Cantor's ideas in an undergraduate analysis class.  At the risk of regurgitating basic concepts, some common sets and their notation as introduced in this class are listed below.  Note that I will be using Robert André's "Axioms and Set Theory" [4] as a reference, and will include 0 in the natural numbers as was done in this text:

\renewcommand{\arraystretch}{1.5}
\begin{table}[hbt!]
 \caption{Common Sets}
  \centering
  \begin{tabular}{lcl}
    \toprule
    Name     & Variable     & Example \\
    \midrule

    The Natural Numbers (Zero Included)   &   $\mathbb{N}$    & $\{ 0, 1, 2, 3, ... \}$ \\ \hline{}

    The Integers   &   $\mathbb{Z}$    & $\{ ..., -3, -2, -1, 0, 1, 2, 3, ... \}$ \\ \hline{}

    The Rational Numbers   &   $\mathbb{Q}$    & Any number representable as a fraction: $1/3$, $10/47$, $101/51947$, etc. \\ \hline{}

    The Real Numbers   &   $\mathbb{R}$    & Any non-imaginary in the continuous interval $(-\infty, \infty)$ \\

    \bottomrule
  \end{tabular}
  \label{tab:table}
\end{table}

Each set listed in Table 1 contains an infinite number of elements.  This is somewhat obvious, especially in the case of the real numbers.  However, when thinking about these sets and their relationships, a more formal definition of infinity is incredibly helpful.

\subsection{Definition of an Infinite Set}
As defined by Richard Dedekind, who at times was a close confidant of Cantor and who was also integral in the formation of Set Theory, an infinite set is a set $S$ in which there exists a one-to-one mapping from $S$ onto a proper subset of itself [4].  It follows that a finite set can be defined as any set that is not infinite.

This definition is ingeniously concise.  However, before utilizing it, some basic ideas about mappings (also known as functions) should be established.

\subsubsection{A Note on Functions}
A function is an operation that, given a set of inputs, yields a set of outputs.  The formal definition of a function is more complicated, but that is beyond the scope of these notes.  What {\em is} important is that the function required by this definition of infinity is defined as being "one-to-one".  This means that each input to the function is associated with a unique output (for example, if $f$ is a function acting on inputs $a$ and $b$, then $f(a) = f(b)$ implies that $a = b$).

It is also important in this definition that the specified function maps a set of inputs, $S$, to a subset of $S$.  For example, the positive even integers ($\{2, 4, 6, ...\}$) are a subset of the natural numbers ($\{0, 1, 2, ...\}$) because each even integer is also natural number, but the opposite is not true.  Common notation used to express this concept is $A \subset B$, meaning $A$ is a subset of $B$.

\subsubsection{Definition Summary}
The primary idea behind this definition is that the only way to take a part of something, and end up with something of the same size, is for the whole to be infinite.  Per Dedekind's definition, if a one-to-one mapping exists between a set $S$ and $S* \subset S$, then $S$ is infinite.  If the mapping is truly one-to-one, then $S$ and $S*$ must each have unique elements that are in one-to-one correspondence.  However, $S* \subset S$ indicates that all elements of $S*$ are in $S$, but the opposite is not true.  The only way for a one-to-one correspondence between $S$ and $S*$ to be possible is if $S$ is infinite.

\subsubsection{$\mathbb{N}$ is Infinite}
This idea is more easily represented by an example.  It can be shown that $\mathbb{N}$ is infinite using the set of positive even integers with zero added (Let $E = \{0, 2, 4, 6, ...\}$).  There is a one-to-one mapping between $\mathbb{N}$ and $E$ given by the following [4]:

\begin{gather*}
f(n) = 2n
\end{gather*}

It is easy to see that for each $n \in \mathbb{N}$ this function maps $n$ to a unique number in $E$ with one-to-one correspondence.  Additionally, it can be seen that every member of $E$ is a member of $\mathbb{N}$, but all positive odd integers are in $\mathbb{N}$ and not in $E$. Therefore, $\mathbb{N}$  must be infinite.

\subsection{Countable Infinity}
A set $S$ is said to be countably infinite if there exists a one-to-one mapping between $S$ and the natural numbers, $\mathbb{N}$ [4].  This implies the existence of "larger" - uncountably infinite - sets, which will be discussed in the next section.  However, it is first interesting to explore the sets that are in one-to-one correspondence with $\mathbb{N}$ and how this can be shown.

\subsubsection{$\mathbb{Z}$ is Countable}
It may be surprising that the set of all integers is the same "size" as the natural numbers given that $\mathbb{Z}$ contains infinitely many negative numbers that are not in $\mathbb{N}$.  However, the following function can be used to show that there is actually a one-to-one mapping between $\mathbb{Z}$ and $\mathbb{N}$ [4]:

\begin{gather*}
f(n) = \begin{cases}
      0 & n = 0 \\
      -\frac{n+1}{2} & n \text{ is odd} \\
      \frac{n}{2} & n \text{ is even}
   \end{cases}
\end{gather*}

As such, $\mathbb{Z}$ is a countably infinite set.

\subsubsection{$\mathbb{N} \times \mathbb{Z}$ is Countable}
To one who is unfamiliar with set theory, $\mathbb{N} \times \mathbb{Z}$ may seem like odd notation.  This "set product" is essentially the set of grouped pairs $(n, z)$ where $n \in \mathbb{N}$ and $z \in \mathbb{Z}$.  It may be equally surprising that a one-to-one mapping exists between $\mathbb{N} \times \mathbb{Z}$ and $\mathbb{Z} - \{0\}$.

Before exploring this mapping, it should be noted that any mapping that is one-to-one with $\mathbb{Z} - {0}$ can be re-written to be one-to-one with $\mathbb{Z}$ via a simple shift in the function.  For example, $\mathbb{Z} - \{0\}$ can be mapped in one-to-one correspondence with $\mathbb{Z}$ via the following function:

\begin{gather*}
f(z) = \begin{cases}
      z & z < 0 \\
      z - 1 & z > 0
   \end{cases}
\end{gather*}

It is clear that 0 is not part of the input, but all integers (including zero) can be produced by this function.  It is also clear that each input yields a unique output.  Thus, any input that can be mapped to $\mathbb{Z} - \{0\}$ with one-to-one correspondence can be mapped to $\mathbb{Z}$, and by extension $\mathbb{N}$ with a one-to-one mapping.

Getting back to the problem at hand, the following function maps $\mathbb{N} \times \mathbb{Z}$ to $\mathbb{Z} - \{0\}$ with a one-to-one association between the two sets [4]:

\begin{gather*}
f(n, z) = 2^{n} (2z - 1)
\end{gather*}

Proving that this is a one-to-one function is beyond the scope of these notes, and a full proof can be found in [4]. As such, $\mathbb{N} \times \mathbb{Z}$ is a countably infinite set.

It can be a fun thought experiment, however, to stress test this with a series of examples:

$f(0, 0) = -1$, $f(1, 0) = -2$, $f(2, 0) = -4$ ...\\
$f(0, 1) = 1$, $f(0, 2) = 3$, $f(0, 3) = 5$ ...\\
$f(0, -1) = -3$, $f(0, -2) = -5$, $f(0, -3) = -7$ ...\\
$f(1, 1) = 2$, $f(1, 2) = 6$, $f(1, 3) = 10$ ...\\
$f(1, -1) = -6$, $f(1, -2) = -10$, $f(1, -3) = -14$ ...\\
$f(2, 1) = 4$, $f(2, 2) = 12$, $f(2, 3) = 20$ ...\\
$f(2, -1) = -12$, $f(2, -2) = -20$, $f(2, -3) = -28$ ...\\

As the number of terms increases, a pattern begins to form.  And, it becomes fascinating to explore how the function, $f(n, z)$ changes as $n$ and $z$ vary.  Most importantly, it becomes clear through these patterns that each input $(n, z)$ seems to produce a unique output.

It also interesting to note that by extending this proof it can be shown that any product of $\mathbb{N}$, i.e. $\mathbb{N} \times \mathbb{N} \times \mathbb{N} \times ... $, is also countably infinite.  This proof is again explained in [4] and beyond the scope of these notes.

\subsubsection{$\mathbb{Q}$ is Countable}
As mentioned in the table above, $\mathbb{Q}$ is the set of rational numbers - or, the set of all numbers that have an exact representation as a fraction of integers.  For example, any decimal that terminates can be written as a fraction ($0.1 = \frac{1}{10}$, $0.01 = \frac{1}{100}$, $0.001 = \frac{1}{1000}$, and so on, and so on...).  However, it is worth mentioning that there are many non-terminating decimals that are rational, the most straight forward example being $\frac{1}{3} = 0.33333333...$.

Proving the "countability" of the rational numbers begins with construction of a function that maps  $\mathbb{Q}$ to $\mathbb{N} \times (\mathbb{Z} - \{0\})$ with one-to-one association.  It is intuitive that $\mathbb{N} \times (\mathbb{Z} - \{0\})$ is countable since $\mathbb{N} \times \mathbb{Z}$ was shown to be countable and it was also shown that a simple shift can be used to define a one-to-one mapping between $\mathbb{Z} - \{0\}$ and $\mathbb{Z}$.  Thus, if $\mathbb{Q}$ is in one-to-one relation with $\mathbb{N} \times (\mathbb{Z} - \{0\})$, it is in one-to-one correlation with $\mathbb{Z}$ and $\mathbb{N}$, and is therefore countably infinite.  The following function maps $\mathbb{Q}$ to $\mathbb{N} \times (\mathbb{Z} - \{0\})$ with a one-to-one association between the two sets [4]:

\begin{gather*}
f(n, z) = \frac{n}{z}
\end{gather*}

\subsection{Uncountable Infinity}
As previously mentioned, the concept of countable infinity implies something larger.  Before discussing the uncountably infinite, it is helpful to define how the size of a set is described.

\subsubsection{Cardinality}
The size of a set is called its "cardinality" (i.e. the number of elements contained in it).  Standard notation for the cardinality of a set is $|\bullet|$.  For example, the cardinality of a set with two elements is two (ex. $|\{1, 3\}| = 2$).

\subsubsection{Definition of Uncountable Infinity}
Uncountable sets can be simply defined as "those infinite sets which are not countable" [4].

\subsubsection{$\mathbb{R}$ is Uncountable}
A common proof that the real numbers are uncountable is a proof by contradiction.  It is clear that the real numbers are infinite.  A quick confirmation

\subsection{Final Thoughts}
There is so much more to this.

\begin{gather*}
|\mathbb{R}| = 2^{|\mathbb{N}|}
\end{gather*}

\section{Why the Cantor Set is the Coolest Thing in Mathematics}
Placeholder

\section{Conclusion}
Infinity in incomprehensible.

\begin{thebibliography}{1}

\bibitem{kour2014real}
Grattan-Guinness, I.
\newblock {\em Towards a Biography of Georg Cantor.}
\newblock Annals of Science, vol. 27, no. 4, 1971, pp. 345–391.
\newblock doi:10.1080/00033797100203837.

\bibitem{kour2014real}
Dauben, Joseph W.
\newblock {\em Georg Cantor and Pope Leo XIII: Mathematics, Theology, and the Infinite.}
\newblock Journal of the History of Ideas, vol. 38, no. 1, 1977, p. 85.
\newblock doi:10.2307/2708842.

\bibitem{1509.02971}
Manin, Yuri I.
\newblock {\em Georg Cantor and his heritage.}
\newblock arXiv, 2002.
\newblock arXiv:math/0209244.

\bibitem{kour2014real}
Robert André.
\newblock {\em Axioms and Set Theory.}
\newblock 2014.
\newblock ISBN:978-0-9938485-0-6.

\bibitem{kour2014real}
Obeng-Denteh, W., et al.
\newblock {\em Cantor's Ternary Set Formula-Basic Approach.}
\newblock British Journal of Mathematics \& Computer Science, vol. 13, no. 1, 2016, pp. 1–6.
\newblock doi:10.9734/bjmcs/2016/21435.

\bibitem{kour2014real}
Athreya, Jayadev S., et al.
\newblock {\em Cantor Set Arithmetic.}
\newblock The American Mathematical Monthly, vol. 126, no. 1, 2019, pp. 4-17.
\newblock doi:10.1080/00029890.2019.1528121.

\end{thebibliography}

\end{document}
